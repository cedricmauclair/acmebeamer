\documentclass[utf8, a4paper]{beamer}

\usepackage[english] {babel}
\usepackage[T1]      {fontenc}
\usepackage{booktabs, ifthen, verbatim}

\usepackage{acmebeamer, lmodern}

\usepackage{tikz}
\usetikzlibrary[shapes.geometric]

%<< [· Theme customs  ·] >>

\leftmargini4mm
\leftmarginii4mm
\leftmarginiii4mm

\definecolor{black}{RGB}{17,11,4}
\definecolor{white}{RGB}{253,252,254}
\definecolor{grey}{gray}{0.35}

\colorlet{red}{red!60!black}
\colorlet{green}{green!90!grey}
\definecolor{blue}{RGB}{0,89,127}

\definecolor{orange}{RGB}{188,90,18}
\definecolor{yellow}{rgb}{0.6,0.6,0.0}

\setbeamercolor{normal text}{fg=black, bg=white}
\setbeamercolor{structure}{fg=blue}
\setbeamercolor{alerted text}{fg=orange}

\setbeamercolor{author}{fg=grey}
\setbeamercolor{institute}{fg=grey}
\setbeamercolor{date}{fg=grey}
\setbeamercolor{framesubtitle}{fg=grey}
\setbeamercolor{subsection in toc}{fg=grey}

\setbeamerfont{author}{size=\large}
\setbeamerfont{date}{size=\large}

\setbeamerfont{alerted text}{series=\bfseries}
\setbeamerfont{example text}{series=\bfseries}
\setbeamerfont{emphasis}{series=\mdseries, shape=\slshape}

\setbeamerfont{title}{series=\bfseries,size=\LARGE}
\setbeamerfont{subtitle}{series=\bfseries,size=\Large}

\setbeamerfont{frametitle}{series=\bfseries,size=\Large}
\setbeamerfont{framesubtitle}{series=\mdseries,size=\large}
\setbeamerfont{section}{series=\bfseries}

\setbeamerfont{block title}{series=\bfseries}
\setbeamerfont{block title alerted}{parent=block title}
\setbeamerfont{block title example}{parent=block title}

\setbeamerfont{section in toc}{size=\normalsize, series=\bfseries}
\setbeamerfont{subsection in toc}{size=\small}

\setbeamertemplate{section in toc}[default]
\setbeamertemplate{items}[default]
\setbeamertemplate{itemize subitem}[acme]
\setbeamertemplate{navigation symbols}{}

\def\slide#1 of#2{\hfill{\normalsize(#1/#2)}\hfillneg}

\renewcommand{\thefootnote}{\fnsymbol{footnote}}

\usebeamerblocks

%>>
%<< [· Handy commands ·] >>

\newcommand*\vfillneg{\vskip0pt plus-1fill}
\newcommand*\hfillneg{\hskip0pt plus-1fill}

\newcommand*\BibTeX{BibTeX}
\newcommand*\cls[1]{\textsf{#1}}
\newcommand*\cs[1]{\texttt{\char`\\#1}}
\newcommand*\meta[1]{\ensuremath{\langle}{\em#1}\ensuremath{\rangle}}
\newcommand*\marg[1]{\texttt{\char`\{#1\char`\}}}
\newcommand*\oarg[1]{\texttt{[#1]}}
\newcommand*\benv[1]{\texttt{\char`\\begin\char`\{#1\char`\}}}
\newcommand*\eenv[1]{\texttt{\char`\\end\char`\{#1\char`\}}}
\newcommand*\env[2]{\benv{#1}\texttt{#2}\eenv{#1}}
\newcommand*\pkg[1]{{\rmfamily\itshape #1}}
\let\type\texttt
\newcommand*\default[1]{\underbar{#1}}

\renewcommand*\LaTeX{LaTeX}
\renewcommand*\LaTeXe{LaTeX2e}
\renewcommand*\TeX{TeX}
\newcommand*\TikZ{Ti{\it k}Z}

\newcommand*\highlight[1]{{\usebeamercolor[fg]{structure}\bfseries#1}}

\let\bf\bfseries
\let\bx\bxseries
\let\sl\slshape
\let\it\itshape
\let\tt\ttfamily
\let\ss\sffamily
\let\sf\rmfamily

\newcommand\applyto[1]{\hfill{\usebeamercolor[fg]{structure}#1}}
\newcommand\stack[2][c]{%
  \let\top\empty\let\bot\empty
  \ifthenelse{\equal{#1}{t}}{\let\cmd\vtop}{}%
  \ifthenelse{\equal{#1}{b}}{\let\cmd\vbox}{}%
  \ifthenelse{\equal{#1}{c}}{%
    \def\cmd{\vbox to0pt}%
    \def\top{\vskip-1ex\vss}\let\bot\vss}{}%
  \setbox0\hbox{\cmd{\top#2\bot}}%
  \hbox to 0pt{\unhbox0}}

%>>

\begin{document}

\title{Documentation \pkg{acmebeamer}}
\author{Cédric Mauclair}
\institute{Onera/DTIM}
\date{February 7, 2011}

\maketitle

\begin{frame}{Overview}
  \tableofcontents
\end{frame}

\section{Minor additions}
%<< [· Title page         ·] >>

\subsection{Title page}

\begin{frame}[fragile]

  \begin{itemize}
  \item Use \cs{maketitle} to typeset title page.
  \item Use \cs{setbeamertemplate} to set title page.
  \item Use \cs{titlegraphic} to set value of \cs{inserttitlegraphic}.
  \end{itemize}

  \vfill
  \begin{itemize}
  \item
\begin{semiverbatim}
\\maketitle\oarg{\meta{options to \\setuptitlepage\oarg{…}}}
          \oarg{\meta{options to \benv{frame}\oarg{…}}}
\end{semiverbatim}
  \item Use \cs{setuptitlepage}\oarg{…} to set options.
    \begin{itemize}
    \item \type{lineoffset}: line v-position of default template;
    \item \type{logo}: \type{institute} or \type{\default{none}};
    \item \type{customlogo}: set value of \cs{logo}.
    \end{itemize}
  \end{itemize}

\end{frame}

%>>
%<< [· Table of contents  ·] >>

\subsection{Table of contents}

\begin{frame}[fragile]

  Use \cs{frametoc} to typeset table of contents. \alert{Beware},
  optional argument comes \alert{after} frametitle (possibly left
  empty).
\begin{semiverbatim}
\\frametoc\marg{\meta{title}}\oarg{\meta{options to \\tableofcontents\oarg{…}}}
\end{semiverbatim}

\end{frame}

%>>
%<< [· Page numbers       ·] >>

\subsection{Page numbers}

\begin{frame}[fragile]

  Use \cs{setuppagenumbers}\oarg{…} to set options.
  \begin{itemize}
  \item \type{style}: \type{plain}, \type{boxed} or
    \type{\default{none}};
  \item \type{custom}: set value directly.
  \end{itemize}

\end{frame}

%>>
%<< [· Head/Footline      ·] >>

\subsection{Head/Footline}

\begin{frame}

  Use \cs{setupheadline}/\cs{setupfootline}\oarg{…} to set options.
  \begin{itemize}
  \item \type{left}/\type{center}/\type{right}:
    (\type{sub})\type{section}, \type{title}, \type{subtitle},
    \type{author}, \type{date} (all short version), \type{page~numbers}
    or \type{\default{empty}};
  \item \{\type{left}|\type{center}|\type{right}\}\type{/subsection}:\\
    \type{\default{text}} (same as \type{…=subsection}), \type{bullets}
    or \type{squares};
  \item \type{frame}: passed to containing \cs{framed}\oarg{…}\marg{…}\\
    (see extra features at end of document).
  \end{itemize}

\end{frame}

%>>
%<< [· Frametitle         ·] >>

\subsection{Frametitle}

\begin{frame}[fragile]

  \begin{itemize}
  \item \type{frametitle} template modified to reclaim space if no
    title. Subtitle use space always, empty or not (may change).
  \item \highlight{New feature: automatic titles.} Enable with
    \cs{acmeframetitlehack} (included at end of \cs{maketitle}).

  \vfill
  \item No title provided: subsection if not empty, or else section, or
    else empty template (with typeset subtitle if provided).
  \item Title provided: normal behaviour.
    \begin{enumerate}[$\star$]
    \item \env{frame}{\marg{\meta{title}}\marg{\meta{subtitle}}…};
    \item
\begin{semiverbatim}
\\begin\marg{frame}
  \\frametitle\marg{\meta{title}}
  \\framesubtitle\marg{\meta{subtitle}}
  …
\\end\marg{frame}
\end{semiverbatim}
    \end{enumerate}
  \end{itemize}

\end{frame}

%>>

\section{Extra features}
%<< [· Super itemize      ·] >>

\subsection{S(uper){\tt itemize} environment}

\begin{frame}[fragile]

  \highlight{New environment: choose level of itemization.}

  \vskip\baselineskip
  Use \env{sitemize}{\oarg{\meta{offset}}…}; can be 1,~2 or~3,
  {\em relative to current nesting level}.

\end{frame}

%>>
%<< [· Blocks on steroids ·] >>

\subsection{Blocks on steroids}

\begin{frame}[fragile]
  {\subsecname \slide 1 of 2}
  {General overview}

  \begin{itemize}
  \item Title becomes optional \alert{but still in braces}; then new
    options
\begin{semiverbatim}
\env{\meta{block}}{\marg{\meta{title}}\oarg{\meta{options}}…}.
\end{semiverbatim}

  \vfill
  \item {\usebeamercolor[fg]{structure}Options}
    \begin{itemize}
    \item \type{width}: set width of block; \applyto{whole block}
    \item \type{color}: \pkg{xcolor} or \pkg{beamer} color name;
      \applyto{whole block}
    \item \type{align}: \type{default}, \type{flushleft},
      \type{flushright},\\
      \hskip\leftmarginii \type{center}, \type{top}, \type{bottom},
      \type{middle} or \verb'{X,Y}'; \applyto{whole block}
    \item \type{title/bodycolor}: \pkg{xcolor} or \pkg{beamer} color
      name; \applyto{title/body}
    \item \type{title/bodystyle}: set style (mostly font);
      \applyto{title/body}
    \item \type{title/bodyalign}: see above; \applyto{title/body}
    \item \type{bodyheight}: set height of body part. \applyto{body}
    \end{itemize}

  \vfill
  \item {\usebeamercolor[fg]{structure}Advanced options}\\
    \type{title/bodycustomframe}, \type{title/bodycustombackground} and
    \type{title/bodyoptions}, all three passed to \cs{framed}.
  \end{itemize}

\end{frame}


\begin{frame}
  {\subsecname \slide 2 of 2}
  {New and customized blocks}

  \begin{itemize}
  \item Create with \cs{defineblock}(\type s)%
    \oarg{\meta{name(,…)}}\oarg{\meta{options}}
    \begin{itemize}
    \item \highlight{inherit defaults}: use
      \cs{defineblock}\oarg{\meta{name}}\oarg{\meta{overide}};
    \item \highlight{copy a block}: use
      \cs{defineblock}\oarg{\meta{to}}\oarg{\meta{from}};
    \item same with plural form, only several new block names.
    \end{itemize}

  \vfill
  \item Customize with \cs{setupblock}(\type s)%
    \oarg{\meta{name(,…)}}\oarg{\meta{options}}
    \begin{itemize}
    \item \highlight{set all}: use \cs{setupblocks}\oarg{\meta{options}};
    \item \highlight{set specific block(s)}: use\\
      \cs{setupblock}(\type s)\oarg{\meta{name(,…)}}\oarg{\meta{options}}.
    \end{itemize}
  \end{itemize}

  \vfill
  See previous page for available options.

\end{frame}

%>>
%<< [· One more thing     ·] >>

\subsection{One more thing}

\begin{frame}[fragile]
  {\subsecname \slide 1 of 2}
  {Welcome \type{framed} \& \type{framedtext} from \pkg{acmetoolbox}}

  \begin{itemize}
  \item Frame small snippets with
    \stack[t]{%
      \cs{framed}\oarg{\meta{options}}\marg{…} \&\\
      \cs{inframe}\oarg{\meta{options}}\marg{…}}
    \begin{itemize}
    \item \type{width}, \type{height}\footnote{Can be preceded by
        \{\type{min}|\type{max}\},
        e.g.~\type{maxwidth=0.8\cs{textwidth}}.}: dimensions of the box;
    \item \type{frame}\footnote{Can be preceded by
        \{\type{left}|\type{right}|\type{top}|\type{bottom}\},
        e.g.~\type{rightframe=on}.}: \type{on} or \type{off};
    \item \type{margin}\footnotemark[2]: dimension {\em outside} the
      box, contribute to \type{width}/\type{height};
    \item \type{offset}\footnotemark[2]: dimension {\em inside} the box;
    \item \type{align}: \type{default}, \type{flushleft},
      \type{flushright}, \type{center}, \type{top}, \type{bottom},\\
      \hskip\leftmarginii \type{middle} or \verb'{X,Y}';
    \item \type{left}/\type{right}/\type{top}/\type{bottom}: added
      material around content;
    \item \type{customframe}: custom \TikZ\ commands to draw frame;
    \item \type{custombackground}: custom \TikZ\ commands to draw
      background;
    \item \type{tikzoptions}: passed to \type{tikzpicture} environment.
    \end{itemize}

  \vfill
  \item Frame larger chunks with
    \env{framedtext}{\oarg{\meta{options}}…}\\
    Same options as \cs{framed}, only \type{width} defaults to
    \verb'\textwidth'.
  \end{itemize}

\end{frame}


\begin{frame}[fragile]
  {\subsecname \slide 2 of 2}
  {Highlight text and pieces of equations conveniently}

  \begin{itemize}
  \item Use \cs{hl<\meta{spec}>\oarg{\meta{options}}\marg{\meta{text}}}
    to mark text.
  \item Use \cs{placehlmark\oarg{\meta{name}}\marg{\meta{text}}} to
    place a \type{name}d mark.
  \item Use \cs{hlmarks\oarg{\meta{options}}\marg{\meta{(n1) (n2) …}}}
    to highlight marks named \type{n1}, \type{n2},~…
  \end{itemize}

\end{frame}

%>>


\end{document}
